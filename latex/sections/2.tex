2. In order to increase the difficulty of breaking your cryptosystem, you decide to encipher your messages using a Hill 2-cipher by first applying the matrix $\begin{pmatrix} 3&11\\4&15 \end{pmatrix}$ working modulo 26 and then applying the matrix $\begin{pmatrix} 10&15\\5&9 \end{pmatrix} $ working modulo 29.

Thus, while your plaintexts are in the usual 26 letter alphabet, your ciphertexts will be in the alphabet with 0-25 as usual and blank=26, ?=27, and !=28.

\begin{enumerate}[label=\alph*)]
  \item Encipher the message “SEND”.
  \item Describe how to decipher a ciphertext by applying two matrices in succession, and decipher “ZMOY”.
  \item Under what conditions is a matrix with entries module 29 invertible modulo 29?
\end{enumerate}

(a) ?CVK

(b) STOP

(c) Using Fermat's Little Theorem, where $p$ is prime, $a^{p-1}\equiv 1 \pmod{p}$
